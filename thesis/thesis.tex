% !TeX encoding = UTF-8
% !TeX program = pdflatex
% !TeX spellcheck = en_US

\documentclass[LaM,binding=0.6cm]{./packages/sapthesis/sapthesis}

\usepackage{microtype}

\usepackage{hyperref}
\hypersetup{pdftitle={Robustness of Deep Neural Networks Using Trainable Activation Functions},pdfauthor={Federico Peconi}}

% Remove in a normal thesis
\usepackage{lipsum}
\usepackage{curve2e}
\definecolor{gray}{gray}{0.4}
\newcommand{\bs}{\textbackslash}

% Commands for the titlepage
\title{Robustness Of Deep Neural Networks \\Using Trainable Activation Functions}
\author{Federico Peconi}
\IDnumber{1823570}
\course{Computer Science}
\courseorganizer{Computer Science - Informatica LM-18}
\AcademicYear{2019/2020}
\copyyear{2020}
\advisor{Prof. Simone Scardapane}
%\advisor{Dr. Nome Cognome}
%\coadvisor{Dr. Nome Cognome}
\authoremail{peconi.1823570@studenti.uniroma1.it}

\examdate{Something October 2020}
\examiner{Prof. Nome Cognome}
\examiner{Prof. Nome Cognome}
\examiner{Dr. Nome Cognome}
\versiondate{\today}



\begin{document}

\frontmatter

\maketitle
\dedication{Dedicated to\\ Donald Knuth}

\begin{abstract}
This document is an example which shows the main features of
the \LaTeXe\ class \texttt{sapthesis.cls} developed by 
with the help of GuIT (Gruppo Utilizzatori Italiani di \TeX).
\end{abstract}

\begin{acknowledgments}
Ho deciso di scrivere i ringraziamenti in italiano
per dimostrare la mia gratitudine verso i membri
del GuIT, il Gruppo Utilizzatori Italiani di \TeX, e, in particolare,
verso il prof. Enrico Gregorio.
\end{acknowledgments}

\tableofcontents

% Do not use the starred version of the chapter command!
%\chapter{Capitolo non numerato}





\mainmatter

\chapter{Introduction}

    \section{Intriguing Properties of Neural Networks}

        Here we informally state the problem of adversarial attacks in ML models 
        especially wrt to Neural Networks.
        Why is it of fundamental importance for the progress of the field from
        both practical (nns cant yet be deployable in critical scenarios for such reasons ) 
        and theoretical (Madry arguments around interpertability and robustness)
        perspectives

    \section{Smooth Activation Functions and Robustness}

        Recently a link has been proposed between activation functions and the robustness of
        Neural Networks (Smooth Adversarial Training). In particular, authors showed how they managed
        to improve the robustness by replacing the traditional Rectified Linear Units activation functions
        with smoother alternatives such as ELUs, SWISH, PReLUs
        
        Building up from this result we thought we could find benefits by laveraging recently proposed smooth
        trainable activation functions called Kernel Based Activation Functions (Scardapane et al.), 
        which already showed great results in standard tasks, in the context of adversarial attacks.
        
    \section{Structure of the Thesis}

        Description of the remaining chapters 


\chapter{Fundamentals}

    \section{Deep Neural Networks}
        
        Broadly speaking, the field of Machine Learning is the summa of any algorithmic methodology whose aim is to automatically find meaningful patterns inside data without
        being explicitly programmed on how to do it. Well known examples are: Search Trees (ref.), Support Vector Machines (ref. ), Clustering (ref.) and, more recently, 
        Neural Networks (ref. ). During the last two decades Neural Networks gained a lot of attention for their outstanding performances in different tasks like
        image classification (ref. ImageNet, over human level), speech and audio processing(ref ).

        \subsection{Definition}
            Neural Networks are often used in the context of Supervised Learning where the objective is to model a function $ \hat{f}:X \mapsto Y$

        \subsection{Training}

        \subsection{CNNs: Convolutional Neural Networks}

        \subsection{From Neural Networks to Deep Neural Networks}
        

    \section{Adversarial Examples Theory}

        Formal definition of what is an adversarial attacks plus currently well known attacks

    \section{Defenses}

        Review of the literature on defenses to improve robustness: provable robustness, adversarial training

    \section{Kernel Based Activation Functions}


\chapter{Related Works}

    \section{K-Winners Take All}
    \section{Smooth Adversarial Training}


\chapter{Solution Approach}
 
    Comparing the activations's distributions for different activation functions (ReLU, KWTA, Kafs) 
    seem to suggest Kafs might be good candidates to improve model robustness 

    \section{Lipschitz Constant Approach}

        On the limitations of current Lipischitz-Constant based approaches especially when involving Kafs

    \section{Fast is Better than Free Adversarial Training}

        Adversarial training (Madry et al.) and current methods to improve the efficiency (Fast is better than free)


\chapter{Evaluation}

    \section{VGG Inspired Architectures Results}

    \section{Explofing Gradients with KafNets}

        The exploding gradients problem with KafResNet, why is it happening? (still to clarify)

    \section{ResNet20 Inspired Architectures Results}


\chapter{Future Works}

    Different Kernels, resolve the exploding gradient problem and scale to ImageNet
    Perform more adaptive attacks to assess the robustness of kafresnets as is the current standard (Carlini et al.)


\chapter{Conclusions}

    This thesis tries to add to the bag of evidences in literature that smoother architectures might benefit improvements in adversarial resiliency
    
    
\appendix


\backmatter
% bibliography
%\cleardoublepage
%\phantomsection
%\bibliographystyle{sapthesis} % BibTeX style
%\bibliography{bibliography} % BibTeX database without .bib extension

\end{document}
